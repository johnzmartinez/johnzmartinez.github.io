\documentclass [11pt, letterpaper] {amsart}
\setlength{\pdfpageheight}{11in}
\setlength{\pdfpagewidth}{8.5in}
\setlength{\topmargin}{-.8in}
\setlength{\topskip}{0in}
\setlength{\textheight}{9.9in}
\setlength{\textwidth}{7.10in}
\setlength{\oddsidemargin}{-.325in}
\setlength{\evensidemargin}{-.325in}
\renewcommand{\theenumi}{\alph{enumi}}
\renewcommand{\labelenumi}{\theenumi.}
\setcounter{secnumdepth}{-1}
\usepackage{amssymb}
%\usepackage{wasysym}
\input xy
\xyoption{all}

\usepackage{graphicx}
\graphicspath{ {images/} }

\theoremstyle{plain}
\newtheorem{exer}{}
\newtheorem{prop}{Proposition}

\theoremstyle{definition}
\newtheorem{exam}{}
\renewcommand{\a}{\mathbf{a}}
\providecommand{\abs}[1]{\lvert#1\rvert}
\renewcommand{\b}{\mathbf{b}}
\newcommand{\B}{\mathcal{B}}
\newcommand{\C}{\mathcal{C}}
\newcommand{\CC}{\mathbb{C}}
\newcommand{\e}{\mathbf{e}}
\renewcommand{\P}{\mathbb{P}}
\newcommand{\R}{\mathbb{R}}
\DeclareMathOperator{\PP}{P}
\DeclareMathOperator{\Proj}{Proj}
%\DeclareMathOperator{\Col}{Col}
\DeclareMathOperator{\Nul}{Nul}
\DeclareMathOperator{\Ran}{Ran}
\DeclareMathOperator{\Span}{Span}
\renewcommand{\v}{\mathbf{v}}
\newcommand{\x}{\mathbf{x}}
\newcommand{\y}{\mathbf{y}}
\newcommand{\z}{\mathbf{0}}

\usepackage[pdftex]{graphicx}
\usepackage{amsmath, amssymb, float}
\usepackage{graphicx}
\usepackage{mathtools}
%\usepackage{subfig}
%\usepackage{pstricks, pst-plot}
%\usepackage{times}
\usepackage{url}
\usepackage{enumerate}
%\newcommand{\HRule}{\rule{\linewidth}{0.1mm}}
\newcommand{\be}{\begin{enumerate}}      
\newcommand{\bea}{\begin{enumerate}[(a)]}
\newcommand{\ee}{\end{enumerate}}


\pagestyle{plain}
\title{Solutions to Homework \#36, MATH 10A, Fall 2018\\ with Professor Stankova
}

\begin{document}
\maketitle

\vspace{0.05in}
\\
We are given the following data on the concentration of HIV in the blood of 15 individuals before and 6 months after a specific treatment for HIV infection.

\smallskip
{\small\begin{center}\begin{tabular}{|c|c|c||c|c|c||c|c|c|}\hline
Patient&Before&After&Patient&Before&After&Patient&Before&After\\\hline
1&7.4&3.7&6&4.3&2.1&11&4.7&2.3\\\hline
2&5.1&2.6&7&5.1&2.6&12&4.7&2.4\\\hline
3&6.9&3.4&8&2.9&1.5&13&9.3&4.6\\\hline
4&7.2&3.6&9&7.2&3.6&14&4.5&2.2\\\hline
5&1.4&0.7&10&3.5&1.7&15&6.0&3.0\\\hline
\end{tabular}
\end{center}}

\begin{exer}
 Plot the data for the first 5 patients.
\end{exer}

\begin{proof}[Answer]
We want to predict the after-treatment concentration of HIV in the blood. The input variable is the before-treatment HIV concentration.\\
\includegraphics[scale=0.7]{plotHW36.png}
\end{proof}


\vspace{0.05in}



\begin{exer}
Write the matrix-vector form of a system leading to the best-fitting line $y=ax+b$ for the first:  (a) 2 patients; \quad (b) 3 patients; \quad (c) 5 patients; \quad (d) (Challenge) all 15 patients.
\end{exer}

\begin{proof}[Answer]
The matrix-vector form will be represented as $A^{T}A\vec{x}=A^{T}\vec{b}$ where $\vec{x} = \left(\begin{matrix}
                    a\\
                    b
                  \end{matrix}\right)$ and: \\
(a) $A = \left(\begin{matrix}
                    7.4 & 1 \\
                    5.1 & 1
                  \end{matrix}\right)$ and  $\vec{b}=\left(\begin{matrix}
                    3.7\\
                    2.6
                  \end{matrix}\right)$\\
(b) $A = \left(\begin{matrix}
                    7.4 & 1 \\
                    5.1 & 1 \\
                    6.9 & 1
                  \end{matrix}\right)$ and  $\vec{b}=\left(\begin{matrix}
                    3.7\\
                    2.6\\
                    3.4
                  \end{matrix}\right)$\\
(c) $A = \left(\begin{matrix}
                    7.4 & 1 \\
                    5.1 & 1 \\
                    6.9 & 1\\
                    7.2 & 1\\
                    1.4 & 1
                  \end{matrix}\right)$ and  $\vec{b}=\left(\begin{matrix}
                    3.7\\
                    2.6\\
                    3.4\\
                    3.6\\
                    0.7\\
                  \end{matrix}\right)$\\
(d) $A = \left(\begin{matrix}
                    7.4& 1 \\
                    5.1 & 1 \\
                    6.9 & 1\\
                    7.2 & 1\\
                    1.4 & 1\\
                    4.3 &1 \\
                    5.1 & 1\\
                    2.9 & 1\\
                    7.2& 1\\
                    3.5& 1\\
                    4.7& 1\\
                    4.7& 1\\
                    9.3& 1\\
                    4.5& 1\\
                    6.0 &1
                  \end{matrix}\right)$ and  $\vec{b}=\left(\begin{matrix}
                    3.7\\
                    2.6\\
                    3.4\\
                    3.6\\
                    0.7\\
                    2.1\\
                    2.6\\
                    1.5\\
                    3.6\\
                    1.7\\
                    2.3\\
                    2.4\\
                    4.6\\
                    2.2\\
                    3.0
                  \end{matrix}\right)$

\end{proof}





\vspace{0.05in}
\begin{exer}
Solve the systems in Problem~2(a)(b)(c). Draw the three best-fitting lines on your plot. For an extra challenge, solve the system for 15 patients in Problem~2(d).
\end{exer}

\begin{proof}[Answer]
We solve the different systems in question by Gaussian elimination and those are the results that we find for the different systems: \\
(a) $\left(\begin{matrix}
                    a\\
                    b
                  \end{matrix}\right) = \left(\begin{matrix}
                    0.47826087\\
                    0.16086957
                  \end{matrix}\right) $
\\(b) $\left(\begin{matrix}
                    a\\
                    b
                  \end{matrix}\right) = \left(\begin{matrix}
                    0.46924829\\
                    0.19886105
                  \end{matrix}\right) $
\\(c) $\left(\begin{matrix}
                    a\\
                    b
                  \end{matrix}\right) = \left(\begin{matrix}
                    0.4964539\\
                    0.01985816
                  \end{matrix}\right) $
\\(d) $\left(\begin{matrix}
                    a\\
                    b
                  \end{matrix}\right) = \left(\begin{matrix}
                    0.49589277\\
                    0.01529332
                  \end{matrix}\right) $
                  
                  \includegraphics[scale=0.4]{plotHW36-2.png}


\end{proof}





\vspace{0.05in}
\begin{exer} 
For any line $y=f(x)=ax+b$, the \textit{residual} of a data point $(x_i,y_i)$ is the difference $y_i-f(x_i)$ between the actual value $y_i$ and what the line would have predicted, $f(x_i)=ax_i+b$. We can calculate the error of a line by adding up the residuals:

\centerline{$E= (y_1-f(x_1))+(y_2-f(x_2))+\cdots+(x_n-f(x_n))$,} 
\noindent or we can add up the \textit{squares} of the residuals and then take the square root of that: \\
$S=\sqrt{(y_1-f(x_1))^2+(y_2-f(x_2))^2+\cdots+(x_n-f(x_n))^2}$.
Why can $E$ turn out to be 0 even if the data is very far from the line? What is the significance of $S$ in the least-square method of best-fitting lines? Finally, for each of your best-fitting lines in Problem~3, compute both errors $E$ and $S$.
\end{exer}

\begin{proof}[Answer]
$E$ can turn out to be 0 even if the data is very far from the line because the value of the residual can be positive or negative, thus the errors can cancel each other. \\
We compute $S$ in the least-square method of best-fitting lines to overcome the issue mentioned previously. We are measuring the vertical distance between the best-fitting line and each data point of our training set. We square them, add them together and take the square root to go back to a distance unit.

we compute E and S for the 3 best-fitting lines we obtained in question 3:\\
(a) E = -0.76956521739 and S = 0.284009904244 for 2 patients\\
(b) E = -0.616628701595 and S = 0.298985768326  for 3 patients\\
(c) E = -0.113475177305 and S = 0.155702431971  for 5 patients\\

\end{proof}





\vspace{0.05in}
\begin{exer} 
Use the three lines you found in Problem~3 to make three predictions about the concentration of HIV in the blood of an individual 6 months after the HIV treatment, given that the patient starts with a concentration of 6.0. Compare with the actual data for such a patient. As an extra challenge, use your best-fitting line for 15 patients to make a prediction about this.
\end{exer}

\begin{proof}[Answer]
The equation of the best-fitting line for the different cases is:\\
(a) $y=0.47826087\times x + 0.16086957$. So for an input of $x=6.0$, the corresponding prediction will be: $0.47826087\times 6.0 + 0.16086957 = 3.03$\\
(b) $y=0.46924829\times x + 0.19886105$. So for an input of $x=6.0$, the corresponding prediction will be: $0.46924829\times 6.0 + 0.19886105 = 3.01$\\
(c) $y=0.4964539\times x + 0.01985816$. So for an input of $x=6.0$, the corresponding prediction will be: $0.4964539\times 6.0 + 0.01985816 = 3.00$\\
(d - challenge) $y=0.49589277\times x + 0.01529332$. So for an input of $x=6.0$, the corresponding prediction will be: $0.49589277\times 6.0 + 0.01529332 = 2.99$
\end{proof}





\vspace{0.05in}
\begin{exer}
The shortcut formulas for $a$ and $b$ of the best-fitting line of points $(x_1,y_1),\ldots,(x_n,y_n)$ are:
\newline\centerline{$\displaystyle{a=\frac{\overline{xy}-\overline{x}\,\overline{y}}{\overline{x^2}-\overline{x}^2},\,\,\quad b=\overline{y}-a\overline{x}, \text{ where } \overline{x}=\frac{\sum_{i=1}^nx_i}{n}, \,\, \overline{xy}=\frac{\sum_{i=1}^nx_iy_i}{n}; \,\,\overline{x^2}=\frac{\sum_{i=1}^nx_i^2}{n};}$}
i.e, $\overline{x}$  is the average of the $x_i$'s; $\overline{xy}$ is the average of the products $x_iy_i$, and so on.
Using these formulas, find again the best-fitting lines for the first 2, 3, and 5 patients, and compare with your results in Problem~3. As an extra challenge, do this also for all 15 patients.


\end{exer}

\begin{proof}[Answer]
For the different cases, we have:\\
(a) For 2 patients:  $\overline{xy}= 20.32$, $\overline{x}= 6.25$, $\overline{y}= 3.15$, $\overline{x^{2}}= 40.385$. So $a=\frac{20.32-6.25\times 3.15}{40.385-6.25^2}=0.47826087$ and $b = 3.15 - 0.47826087\times 6.25 = 0.16086956$  \\
(b) For 3 patients:  $\overline{xy}= 21.367$, $\overline{x}= 6.467$, $\overline{y}= 3.23$, $\overline{x^{2}}= 42.793$. So $a=\frac{21.367-6.467\times 3.23}{42.7933-6.467^2}=0.49277654$ and $b = 3.23 - 0.49277654\times 6.467 = 0.04321412$  \\
(c) For 5 patients:  $\overline{xy}= 18.2$, $\overline{x}= 5.6$, $\overline{y}= 2.8$, $\overline{x^{2}}= 36.436$. So $a=\frac{18.2-5.6\times 2.8}{36.436-5.6^2}=0.49645390$ and $b = 2.8 - 0.49645390 \times 5.6 = 0.019858160$  \\
(d) For 15 patients:  $\overline{xy}= 16.152$, $\overline{x}= 5.35$, $\overline{y}= 2.67$, $\overline{x^{2}}= 32.407$. So $a=\frac{16.152-5.35\times 2.67}{32.407-5.35^2}=0.49346017$ and $b = 2.67 - 0.49346017 \times 5.35 = 0.029988091$  \\
Those values are really close to the results found in Problem 3 (the small differences come from approximations made during the computation).
\end{proof}


\vspace{0.05in}
\begin{exer} 
We do not need special methods to find the best-fitting line for the first two patients! Why? Find this line in a third (very simple) way. Compare with your answers for 2 patients above.
\end{exer}

\begin{proof}[Answer]
There is exactly one line joining two distinct points. Its equation can be found this way: \\
We know the equation can be written $y=ax+b$ with the slope a and the intercept b. We know also that this line passes by the points (7.4,3.7) and (5.1,2.6). So the slope a is equal to $a=\frac{3.7-2.6}{7.4-5.1} = 0.47826087 $ that is equal to the value of a found in Problem 3 for 2 patients. The intercept b can be found using one of the two points. We have $3.7 = 0.47826087\times7.4 + b$, so $b=3.7 - 0.47826087\times7.4=0.16086957 $ which is also the value of b found in Problem 3.

\end{proof}





\vspace{0.05in}

    
\vspace{0.05in}
\begin{exer} 

(Challenge) Find the best-fitting \textit{parabola} for the data on patients 12, 13, 14, and 15. \newline (\textit{Hint:} Instead of the $n\times 2$ matrix $A$ whose first column is the input vector $\vec{x}$ and whose second column is the constant vector $\vec{u}$ with all entries equal to 1, use the $n\times 3$ matrix $A$ that has one extra column on the left with entries $x_1^2, x_2^2,\ldots, x_n^2$. 
   
\end{exer}

\begin{proof}[Answer]
Following the hint, we create this extra column in A (concerning only the patients 12 to 15): \\
$A = \left(\begin{matrix}
                    4.7^2 &4.7& 1\\
                    9.3^2 &9.3& 1\\
                    4.5^2 &4.5& 1\\
                    6.0^2 &6.0 &1
                  \end{matrix}\right)$
                  \\and b is still: $\vec{b}=\left(\begin{matrix}
                    2.4\\
                    4.6\\
                    2.2\\
                    3.0
                  \end{matrix}\right)$
            
We follow the same reasoning by solving $\left(\begin{matrix}
                    a\\
                    b
                  \end{matrix}\right) = (A^TA)^{-1}A^T\vec{b} = \left(\begin{matrix}
                    -0.00606322\\
                    0.57439193\\
                    -0.21830052
                  \end{matrix}\right)$ which represents the best-fitting parabola: $f(x) = -0.00606322x^2 + 0.57439193x -0.21830052  $
               
\end{proof}





\vspace{0.05in}







\prob{True \& False}
\begin{enumerate}
    \item (Certain impossibility) $A^TA$  cannot be a symmetric matrix if A  is not square.
                    
\\
    \emph{False. $A^TA$ is always symmetric because if we transpose it is still equal to $A^TA$. This comes from the following formulas: $(AB)^T=B^TA^T$, from which $(A^TA)^T=A^T(A^T)^T= A^TA$}\\\item 
    (Canceling opposites or flip-flopping?) The sum of the residuals of data points from a line is not a good estimate of the fitness of the line, since this sum could be large, yet the data points could be very close to the line.
                    
\\
    \emph{False. The reason why the sum of the residuals of data points is not a good estimate of the fitness of the line is because its value could be 0 even if the points are very far from the line (see previous question 4 of the problem).}\\
    \item 
        (Deep comparison) The error that we want to minimize when finding the least-square best-fitting line for a bunch of data points is analogous to the standard error in a set of data.
                    
\\
    \emph{True. The standard error, though, is also averaging the square distances, and hence we also divide there by n = the size of the data set.}\\\item 
                        (Going "Einstein" to 4 and more dimensions!) The least-square best-fitting line for any number of data points  always exists and is unique essentially because there is a (unique) shortest distance from a point to a plane in any dimensions.
\\
    \emph{False. The formula $\left(\begin{matrix}
                    a\\
                    b
                  \end{matrix}\right) = (A^TA)^{-1}A^T\vec{b}$ is valid only if the matrix $(A^TA)^{-1}$ is invertible. In that case there is only one line. Otherwise, there are an infinite number of solutions for the parameters a and b. But as long as not all $x_i$'s are the same (i.e., there are at least two different input values $x_i$ and $x_j$), the determinant of  $A^TA$ will be $n(\overline{x^2}-(\overline{x})^2)>0$ and we will have a unique solution, because the matrix $A^TA$ will be invertible.}\\\item 
                    
                    
                        If $A=\left(\begin{array}{cc} x_1&1\\x_2&1\\\vdots&\vdots\\ x_n&1\end{array}\right)$   for the inputs  $x_1,x_2,\ldots,x_n$ of our data points,\\
                        \begin{itemize}
                            \item 
                        (Repetition in disguise) The normal equation $$ A^TA\left(\begin{array}{c}a\\b\end{array}\right)=A^T\vec{y}$$  may fail to have a solution for some special choice of different inputs  $x_1,x_2,\ldots,x_n$
\\
    \emph{False. This equation always has a unique solution, as long as not all $x_i$'s are equal}\\\item 
                            (Mandatory "push-ups") We must find the inverse of  $A^TA$ in the normal equation in order to find the coefficients a  and  b  of the best-fitting line.
\\
    \emph{False. We can also solve the system with the augmented matrix using Gaussian Elimination.}\\\item 
                            (Mixing different species!) The entries of the matrix $A^TA$  in the normal equation are  \\$1, \sum_{i=1}^nx_i, \sum_{i=1}^nx_iy_i, \text{ and } \sum_{i=1}^nx_i^2 $
\\
    \emph{False. The matrix $A^TA$ has to be symmetric so the term $\sum x_i$ is repeated across the diagonal of $A^TA$; for that, $A^TA$ contains no $y_i$'s, and hence the term $\sum x_iy_i $ is not part of it.}\\\item 
    
            (Shortcut det) $ \frac{1}{n}\det(A^tA)=\overline{x}^2-\overline{x^2} $, where the definitions of $\overline{x}^2$  and  $\overline{x^2}$   can be found somewhere in this HW assignment.
                    
\\
    \emph{True. We know that $\frac{1}{n}A^TA=\left(\begin{matrix}
                    1 & \overline{x}\\
                    \overline{x} & \overline{x^2}
                  \end{matrix}\right)$ which means that $\frac{1}{n}det(A^TA) =\overline{x}^2-\overline{x^2} $}\\
\end{itemize}                  
\item 
                        (The more, the merrier!) If we use more data points to find the best-fiting line, we may increase the overall error  S  yet still be able to make better predictions about the data.
\\
    \emph{True. If we add more data, S will be bigger but the prediction we can make better predictions by finding a better fitting line.}\\\item 
                        (Are all implications of a theorem necessarily true?) The second super-shortcut formula for a  and b  can be rewritten as $\overline{y}=a\overline{x}+b$ , meaning that the point  $(\overline{x},\overline{y})$  (whose coordinates are the averages of the corresponding coordinates of the data points) lies exactly on the best-fitting line, and this doesn't make sense since it does not always have to happen! 
\\
    \emph{False: Yes, the point $(\overline{x},\overline{y})$ lies exactly on the best-fitting line, and this makes sense numerically since this is the average point of the data, and if any point should lie on the best-fitting line, it should be this average point. }
                
    
    
\end{enumerate}




\end{document}